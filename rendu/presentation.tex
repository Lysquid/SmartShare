\documentclass{beamer}

\usetheme{Madrid}

\title[Smart Share]{Smart Share}
\author[Blazanome]{Romain Benoît, Théo Choné, Ewan Chorynski, Florian Delhon, Théo Gaigé, Amar Hasan Tawfiq, Alix Peigue}
\institute[INSA Lyon]{INSA Lyon}
\date{2024}

\usepackage[T1]{fontenc}
\usepackage[utf8]{inputenc}
\usepackage{listings}
\usepackage{tikz}
\usepackage{xcolor}
\usetikzlibrary{positioning}
\usetikzlibrary{shapes.geometric}

\lstset{%
basicstyle=\tiny\ttfamily,
breaklines = true,
language=C,
numbers=left,
numberstyle=\tiny\color{gray},
keywordstyle=\color{blue}\ttfamily,
stringstyle=\color{red}\ttfamily,
commentstyle=\color{green}\ttfamily,
morecomment=[l][\color{magenta}]{\#},
xleftmargin=2em,frame=single,framexleftmargin=1.5em
}

\newcommand*{\thead}[1]{\multicolumn{1}{|c|}{\bfseries #1}}
\newcommand*{\local}[1]{\lstinline|\_#1|}

\begin{document}

\frame{\titlepage}

\begin{frame}
	\frametitle{Test}
	\framesubtitle{Testtest}

\end{frame}

\section{Première itération}
\begin{frame}
    \frametitle{Architecture globale}
    \includegraphics[width=\textwidth,height=0.8\textheight,keepaspectratio]{archi.png}
\end{frame}

\begin{frame}
    \frametitle{Objectifs}
    \begin{block}{Existant}
        \begin{itemize}
            \item Google Docs: édition sur le navigateur, serveurs propriétaires de Google
            \item LiveShare: restreint à VSCode
            \item CodeTogether: payant, propriétaire
        \end{itemize}
    \end{block}
    \begin{block}{Notre proposition}
        Une solution open-source d'édition de fichiers simultanée fonctionnant sur plusieurs éditeurs.
        \begin{itemize}
            \item Un protocole de communication
            \item Un client exécutable universel à tous les éditeurs de textheight
            \item Une intégration simple dans les éditeurs/IDE
        \end{itemize}
    \end{block}
\end{frame}

\begin{frame}
    \begin{block}{Temps réel}
        \begin{itemize}
            \item Latence faible
            \item Edition simultanée des fichiers (nombreux enjeux abordés en détail dans les slides suivantes)
        \end{itemize}
    \end{block}
    \begin{block}{Interopérabilité}
        \begin{itemize}
            \item Fonctionalités équivalentes sur plusieurs éditeurs
            \item Ajout communautaire d'éditeurs
            \item Facilité d'intégration
        \end{itemize}
    \end{block}
\end{frame}

\begin{frame}
    \frametitle{Architecture globale}

    \begin{block}{Avantages}
        \begin{itemize}
            \item Factoriser la complexité dans le client
            \item Favoriser l'interopérabilité
            \item Faciliter la communication avec le serevur
       \end{itemize}
    \end{block}

    \begin{block}{Inconvénients}
        \begin{itemize}
            \item Deux protocoles à gérer
            \item Deux points de synchronisation
            \item Toujours un peu de logique à coder directement dans l'IDE
        \end{itemize}
    \end{block}
\end{frame}

\begin{frame}
    \frametitle{Choix du langage}
    \framesubtitle{Rust}

    \begin{columns}
        \begin{column}{0.7\textwidth}
            \begin{itemize}
                \item \textit{Fast, Reliable, Productive. Pick Three.}
                \item \textit{Fearless parallelism}
                \item Langage de programmation système
                \item Adapté à la programmation parallèle sans risque
                \item Grand entrain de la part du groupe pour apprendre ce nouveau langage
            \end{itemize}
        \end{column}
        \begin{column}{0.3\textwidth}
            \includegraphics[width=\textwidth,height=0.2\textheight,keepaspectratio]{ferris.png}
        \end{column}
    \end{columns}
\end{frame}
\begin{frame}[fragile]
    \frametitle{Transport de données}
    \framesubtitle{Sérialisation}
    \begin{block}{Sur la ligne}
        \begin{itemize}
            \item Messages sérialisés en JSON
            \item Un message par ligne
        \end{itemize}
    \end{block}
    \begin{exampleblock}{Exemple}
        \begin{lstlisting}
{"action":"update","changes":[{"offset":3413,"delete":0,"text":"i"}]}
{"action":"update","changes":[{"offset":3414,"delete":2,"text":""}]}
{"action":"update","changes":[{"offset":3436,"delete":0,"text":"lstlisting"}]}
{"action":"ack"}
        \end{lstlisting}
    \end{exampleblock}
    \begin{block}{Côté Rust}
        \begin{itemize}
            \item Structures fortement typés
            \item Des Streams pour recevoir
            \item Des Sinks pour envoyer Un message par ligne
        \end{itemize}
    \end{block}
\end{frame}

\begin{frame}
    \frametitle{Solution naïve}

    \begin{block}{Serveur}
        \begin{itemize}
            \item Transmet les changements reçus d'un client vers tous les autres
            \item Ressemble à un simple chat
        \end{itemize}
    \end{block}
    \begin{block}{Client}
        \begin{itemize}
            \item Transmet simplement les messages entre l'IDE et le serveur
            \item Simple boîte au lettre
        \end{itemize}
    \end{block}
    \begin{alertblock}{Ce n'est pas si simple !}
        Gros problèmes lorsque plusieurs personnes écrivent en même temps.
    \end{alertblock}
\end{frame}

\begin{frame}
    \frametitle{Operational Transform (OT) - 1/3}
    \begin{block}{Contexte}
        \begin{itemize}
            \item $\neq$ Git - changements asynchrones $\Rightarrow$ nécessité de \textit{commit}
            \item Google Docs, Etherpad, LiveShare - changements synchrones, temps-réel
            \item Deux stratégies - \textit{Event passing} ou \textit{Differential sync}
        \end{itemize}
    \end{block}

    \begin{block}{Principe}
        \begin{itemize}
            \item Algorithme de type \textit{Event passing}
            \item Chaque client envoie les \textit{events} caractère par caractère au serveur
            \item (Exemples : 'M' inséré à la position 4, suppression d'un caractère à partir de la position 10, ...)
            \item Objectif : maintenir et synchroniser un état cohérent entre les utilisateurs
        \end{itemize}
    \end{block}
\end{frame}

\begin{frame}
    \frametitle{Operational Transform (OT) - 2/3}
    \begin{block}{Problème}
        \begin{itemize}
            \item Si tous les clients exécutent les mêmes \textit{events} dans un ordre différent
            \item $\Rightarrow$ résultats potentiellement différents
            \item Exemple : $a = ins(X, 1)$ et $b = ins(Y, 1) \Rightarrow a \times b \neq b \times a$
        \end{itemize}
    \end{block}

    \begin{block}{Solution : Transformation}
        \begin{itemize}
            \includegraphics[width=\textwidth,height=0.8\textheight,keepaspectratio]{diamond_ot.png}
            \item Fonction de transformation - calculer les nouvelles opérations complémentaires tel que :
            \item $\Rightarrow$ $a \times b' = b \times a'$
            \item Exemple - $a = ins(X, 1)$ et $b = ins(Y, 1) \Rightarrow a' = a$ et $b' = ins(Y, 2)$
        \end{itemize}
    \end{block}
    
    $\Rightarrow$ \textit{Crate} Rust "operational_transform" et structure OperationSeq

\end{frame}

\begin{frame}
    \frametitle{Operational Transform (OT) - 3/3}
    \begin{block}{Application dans SmartShare}
        \begin{itemize}
            \item Décomposition du document en trois ensembles : $A \times X \times Y$
            \item $A$ : état stable, $X$ : changements envoyés, $Y$ : changements non envoyés
            \item Gestion des conflits dans le Client et le Serveur
        \end{itemize}
    \end{block}

    \begin{block}{Serveur}
        \begin{itemize}
            \item Approbation des changements envoyés par le Client
            \item Calcul des nouvelles opérations pour maj du doc Serveur + envoi aux Clients
        \end{itemize}
    \end{block}

    \begin{block}{Client}
        \begin{itemize}
            \item Enregistrement des changements locaux IDE
            \item Soumission de ses changements au Serveur
            \item Application des changements des autres Clients
        \end{itemize}
    \end{block}
\end{frame}

\begin{frame}
    \frametitle{Gestion de projet}
    \framesubtitle{Méthode agile}
    \begin{block}{Organisation générale}
        \begin{itemize}
            \item Projet séparés en tâches : plugins IDE, serveur & client, gestion de conflits, protocoles de communication, ...
            \item 1 sprint par jour avec état d'avancement chaque matin
            \item Document de suivi décrivant les objectifs du jour
        \end{itemize}
    \end{block}
\end{frame}

\begin{frame}
    \frametitle{Conclusion}
    \framesubtitle{Bilan du projet}
    \begin{block}{Fonctionnalités non développées}
        \begin{itemize}
            \item Utilisation en réseau non local
            \item Partage de plusieurs fichiers en même temps
            \item Support d'autres éditeurs de texte
        \end{itemize}
    \end{block}
    \pause
    \begin{block}{Fonctionnalités développées}
        \begin{itemize}
            \item Support de plusieurs IDE sur le même fichier
            \item Synchronisation en temps réel
            \item Gestion des conflits
            \item Visualisation des curseurs
        \end{itemize}
    \end{block}
    \begin{exampleblock}{Exemple}
        Cette présentation a été codée à plusieurs en même temps grâce à Smart Share !
    \end{exampleblock}
\end{frame}


\begin{frame}
eh théo on baise la mère d'Ewan OUI OK

bébou..... ??? Est-ce que tu m'aimes ??
Est-ce que tu me gnoc ?

est-ce que tu me gnoni ??

RATIO
\end{frame}

\end{document}

 